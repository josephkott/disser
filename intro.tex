\chapter*{Introduction}
\addcontentsline{toc}{chapter}{Introduction}
\label{introduction}

Since the 90s of the last century, the Nonlinear Schr\"odinger Equation (NLS) with additional spatial non-autonomous terms has been an object of thorough studies.
Specific interest to this class of equations has been caused by progress in experimental study of Bose--Einstein Condensate (BEC).
Being predicted in 20-ies \cite{Einstein, Bose}, this state of matter was obtained for the first time in 1995 independently in experiments of two groups of researchers \cite{WiemanCornell, Ketterle}.
In 2001 authors of this discovery were awarded the Nobel Prize.

The observation of BEC stimulated experimental and theoretical studies of ultracold cases all over the world.
These studies exhibited that BEC may have numerous practical applications.
Nowadays, it's expected that BEC may be used for the development of the new high-frequency interferometers \cite{GrossZiboldNicklasEsteveOberthaler}, in designing of quantum computers \cite{JakschZoller} and quantum lasers \cite{GuerinRiouGaeblerJosseBouyerAspect}.

It turn out that the dynamics of elongated BEC in so-called mean-field approximation is well described by the Schr\"odinger type equation with spatial non-autonomous terms,
\begin{equation}
	i \Psi_t + \Psi_{xx} - U(x) \Psi + P(x) |\Psi^2| \Psi = 0.
\label{eq:intro-gpe}
\end{equation}
In the context of BEC, equation \eqref{eq:intro-gpe} is called the {\it Gross--Pitaevskii equation} (GPE).
Here $\Psi(t, x)$ is the dimensionless wave function of the condensate cloud, that is assumed to be elongated along the axis $x$.
The function $U(x)$ describes the trap potential that is used to confine BEC, and $P(x)$ corresponds to nonlinear potential (also called {\it pseudopotential}).
The pseudopotential describes the spatial dependence of scattering length that may be non-constant due to various reasons.
Possible experimental techniques to achieve non-constant scattering length involve optical or magnetic Feshbach resonances \cite{PollackDriesJunkerChenCorcovilosHulet, ChinGrimmJulienneTsienga, BauerLetterVoRempeDurr}.
Intervals with positive values of pseudopotential, $P(x) > 0$, correspond to the case of attraction of the condensate particles, while intervals with negative values, $P(x) < 0$, correspond to the interatomic repulsion.
The prototypical examples of $U(x)$ are the harmonic potential $U(x) = Ax^2$ (the magnetic trap), periodic potential $U(x) = A \cos 2x$ (the optical trap), and various types of potential wells.
As examples of the pseudopotential $P(x)$, different functions has been used, including periodic ones.
In the latter case one says that there exists a {\it nonlinear lattice} interacting with the condensate cloud.
In physical experiments nonlinear lattice is created by a resonant action of external magnetic \cite{InouyeAndrewsStengerMiesnerStamperKurnKetterle} or optical \cite{ClarkHaXuChin} field.
The presence of nonlinear lattice may provide additional possibilities for stabilization of the condensate cloud and for exciting of new condensate states.
In order to model a nonlinear lattice in theoretical works the cosine pseudopotential of the form $P(x) = A + B \cos \Omega x$ is traditionally used \cite{SakaguchiMalomed}.

It's also worth noting that similar problems arise in other branches of physics, in particular in nonlinear optics \cite{KartashovMalomedTorner}.
In various optical applications, equation \eqref{eq:intro-gpe} describes the propagation of a light beam in an optical fiber.
In this case, periodic spatial modulation of the Kerr coefficient, related to nonlinear term in \eqref{eq:intro-gpe}, can be induced by inhomogeneous density of resonant nonlinearity-enhancing dopants implanted into the waveguide \cite{HukriedeRundeKip}.
%In this case, periodic spatial modulation of nonlinearity is achieved by adding resonant dopants into the fiber \cite{HukriedeRundeKip}.

For different physical applications the solutions of equation \eqref{eq:intro-gpe} of the special form, so-called {\it stationary localized solutions} ({\it stationary localized modes}, SLMs), play an important role.
Such solutions can be obtained by putting into \eqref{eq:intro-gpe} an ansatz
\begin{equation}
	\Psi(t, x) = u(x) e^{-i \omega t},
\label{eq:intro-ansatz}
\end{equation}
where the function $u(x)$ satisfies the localization conditions of the form:
\begin{equation}
	\lim \limits_{x \to \infty} u(x) = 0.
\label{eq:intro-localization}
\end{equation}
Here $\omega$ is a real parameter that stands for a chemical potential of the condensate.
The profile $u(x)$ of the stationary localized solution is a real-valued function \cite{AlfimovKonotopSalerno}, satisfying the equation
\begin{equation}
	u_{xx} + Q(x) u + P(x) u^3 = 0; \quad Q(x) = \omega - U(x).
\label{eq:intro-stationary}
\end{equation}

It should be noted that not all localized solutions of Eq.~\eqref{eq:intro-stationary} are equally interesting from a physical point of view.
The stability is a critically important property of the localized solutions.
If SLM is unstable, a small perturbation leads to its destruction during the temporal evolution.
Therefore, stable localized solutions are especially valuable from the perspective of physical applications.
So, the analysis of stability is an essential part of the theoretical study of SLMs.

\textbf{Formulation of the problem.}
While studying the dynamics described by equation \eqref{eq:intro-gpe} the following questions naturally arise:
\begin{enumerate}
	\item Is it possible to describe {\it completely all} stationary localized solutions of equation \eqref{eq:intro-gpe} that coexist for under given parameters?
	\item How to identify stable solutions among them?
\end{enumerate}

\textbf{A survey of the current state of the field.}
It's worth noting that in the majority of works devoted to this topic the problem of finding / describing of {\it all} SLMs has not been raised.
Instead, only specific classes of solutions corresponding to particular physical structures has been described, see the comprehensive review \cite{KartashovMalomedTorner}.
At the same time, despite the questions above seem a little bit ``challenging'', the combination of rigorous analytical methods with numerical computations makes it possible to achieve significant progress in this direction.
Let us note some related results.

For equation \eqref{eq:intro-stationary} with potential $U(x)$ of the form of infinite potential well, in the case of repulsive interparticle interactions, $P(x) \equiv -1$, the computational descriptive procedure has been proposed \cite{AlfimovZezyulin}.
This procedure provides a computational evidence and can guarantee the complete description of {\it all} bounded solutions of equation for the given set of parameters.
The proposed method was afterwards generalized to systems of several coupled Gross--Pitaevskii equation, in which the corresponding pseudopotential do not depend on the spatial coordinate \cite{AlfimovBarashenkovFedotovSmirnovZezyulin}.

It was shown, that for the periodic potential $U(x)$ in the case of repulsive interactions of the condensate particles, $P(x) \equiv -1$, there exist sufficient conditions that allow to describe exhaustively {\it all} bounded solutions of equation \eqref{eq:intro-stationary}.
Moreover, it was shown that under these conditions there exist one-to-one correspondence between the bounded solutions and all possible bi-infinite sequences of symbols of some finite alphabet \cite{AlfimovAvramenko}.
Such sequences are called {\it codes of solutions}, and the algorithm of assigning of such codes can be called {\it coding of solutions}.
In the above mentioned paper, the verification of the sufficient conditions was performed by means of numerical computations.
Results of this paper were further extended \cite{AlfimovKizinZezyulin}, specifically: there has been proposed an algorithm that allows to reconstruct numerically the profile of the solution by its symbolic code.

It's also worth to mention the mathematical works of F. Zanolin and co-authors \cite{ZaniniZanolin2018, ZaniniZanolin2012}, in which the existence of some types of solutions in related problems is proved.
Such solutions also can be classified by means of methods of nonlinear dynamics.
Authors of these works use an approach that relies on topological argumentation and differs from the one presented in the dissertation.

\textbf{Relevance of the research topic.}
Generalization of the above mentioned results to the case of non-constant pseudopotential, $P(x) \neq \mathrm{const}$, is an important actual problem.
Application of the ``coding approach'' to the Gross-Pitaevskii equation with periodic pseudopotential yields  a classification of nonlinear stationary states in BEC in nonlinear lattice.
This classification opens up the possibility of experimental finding of new, previously unknown stable stationary states.

\textbf{Tasks and objectives of the study.}
The main object of the study in the dissertation is the set of stationary solutions of one-dimensional Gross -- Pitaevskii equation \eqref{eq:intro-gpe} with {\it periodic pseudopotential}.
Tasks and objectives of the study can be formulated as follows:
\begin{enumerate}
	\item To formulate sufficient conditions that allow to generalize the method of coding of SLMs \cite{AlfimovAvramenko} to the case of periodic pseudopotential; to specify the ways of verification of these conditions (analytically or with numerical computations).
	\item To study the set of stationary solutions of equation \eqref{eq:intro-gpe} with periodic pseudopotential in the case when the trapping potential can be neglected, $U(x) \equiv 0$.
	\item For the case of harmonic potential well, $U(x) = A x^2$, to investigate the effect of periodic pseudopotential on the structure of the set of stationary localized solutions and their stability.
\end{enumerate}

\textbf{Scientific novelty.}
In the thesis, a number of exact statements about regular and singular solutions of equation \eqref{eq:intro-stationary} are proved.
The conditions that ensure the existence of singular solutions, or their absence are formulated.
In particular, it was shown that if the pseudopotential is negative at least at one point $x_0$, $P(x_0) < 0$, then there exist two one-parametric families of solutions which tend to infinity at this point $x_0$.
Asymptotic expansions for these families are given.

The method of excluding of singular solutions was further developed.
The dissertation proposes sufficient conditions for existence of one-to-one correspondence between regular solutions of equation \eqref{eq:intro-stationary} and bi-infinite symbolic sequences over some alphabet. 
In contrast to the previously obtained results \cite{AlfimovAvramenko}, the proposed conditions admit effective numerical verification.
An algorithm of the numerical verification is provided in the dissertation along with its theoretical justification.
This algorithm was implemented in {\tt MATLAB} using {\tt MEX} extension for high performance computing support.

For the case $U(x) \equiv 0$ and cosine periodic pseudopotential of the form $P(x) = A + B \cos 2x$ the set of stationary localized solutions has been studied.
When applying the above-mentioned techniques, the set of SLMs was effectively described, and, eventually, new stable localized solution, named {\it dipole soliton} \cite{LebedevAlfimovMalomed}, was found.
This solution has been previously unknown.

Finally, in the case of harmonic trapping potential, $U(x) = A x^2$, the effect of periodic pseudopotential of the form $P(x) = A + B \cos \Omega x$ on the set of SLMs was studied.
It was shown that in comparison with well-studied case $P(x) = \mathrm{const}$, the set of stationary localized solutions is much richer.
Namely, there exist essentially nonlinear solutions which cannot be predicted by low-amplitude approximation.
The dependence of the SLMs stability on the frequency $\Omega$ of the pseudopotential was studied.
For the pseudopotential with zero mean, $P(x) = B \cos \Omega x$, it was found that the increase of frequency $\Omega$ allows to stabilize low-amplitude solutions, whose counterparts in the model with $P(x) = \mathrm{const}$ are unstable.

\textbf{The highlights of the thesis are:}
\begin{enumerate}
	\item The statements on the presence and absence of singular solutions of equation \eqref{eq:intro-stationary} are proved.
		It is shown that in the case $P(x) > 0$ all solutions of \eqref{eq:intro-stationary} are regular.
		If $P(x)$ is negative at least at one point $x_0$, $P(x_0) < 0$, then there exist two one-parametric families of solutions, which tend to infinity at the point $x_0$.
		The asymptotic description of these families are given.
		In the case $Q(x) < 0$ and $P(x) < 0$, it is shown that all solutions of equation \eqref{eq:intro-stationary} are singular.
	\item For equation \eqref{eq:intro-stationary} sufficient conditions for coding of regular solutions are formulated.
		An effective algorithm for their numerical verification is presented.
	\item For the case $U(x) \equiv 0$, $P(x) = A + \cos 2x$ the set of SLMs of Eq.~\eqref{eq:intro-gpe} are described.
		This study reveals new stable localized solution, named {\it dipole soliton}.
	\item It is shown that the model that includes both trapping harmonic potential $U(x) = A x^2$ and the nonlinear lattice admits new classes of SLMs, in comparison with the case when the nonlinear lattice is not taken into account.
		For the periodic pseudopotential with zero mean, it is shown that increasing of the frequency of pseudopotential can stabilize low-amplitude localized solutions.
\end{enumerate}

\textbf{Confidence level and approbation of the results.}
The Gross -- Pitaevskii model is a classical model of physics of ultra-cold temperatures and its confidence is beyond any doubt.
SLMs in this model correspond to localized stationary solutions of the Gross--Pitaevskii equation.
In the thesis, SLMs are constructed, and their stability is investigated numerically.
Numerical computation of SLMs is performed by means of standard numerical methods for ODEs with controlled accuracy.
The analysis of stability of SLMs is fulfilled by means of the spectral method which is generally recognized for the similar problems.
Results of the stability analysis are verified by solution of the time-dependent Gross--Pitaevskii equation employing a conservative finite-difference scheme.
The key findings of the thesis were reported at various scientific seminars and conferences, including:
\begin{enumerate}
	\item <<Фундаментальная математика и ее приложения в естествознании>>, BSU, Ufa, September, 2015.
	\item ``Dynamics, Bifurcations and Chaos III'', Lobachevsky State University of Nizhni Novgorod, Nizhni Novgorod, July, 2016.
	\item ``Complex Analysis, Mathematical Physics and Nonlinear Equations'', Bashkortostan, Bannoe Lake, March, 2018.
	\item ``Nonlinear Phenomena in Bose Condensates and Optical Systems'', Tashkent, Uzbekistan, August, 2018.
	\item ``Complex Analysis, Mathematical Physics and Nonlinear Equations'', Bashkortostan, Bannoe Lake, March, 2019.
	\item ``Complex Analysis, Mathematical Physics and Nonlinear Equations'', Bashkortostan, Bannoe Lake, March, 2021.
\end{enumerate}

\textbf{Publications.}
Materials of the thesis were presented in 9 publications, among them there are 3 articles in peer-reviewed journals \cite{AlfimovLebedev, LebedevAlfimovMalomed, AlfimovGegelLebedevMalomedZezyulin}, and 6 conference proceedings \cite{Ufa2015, NizhniNovgorod2016, Bannoe2018, Tashkent2018, LebedevShipitsynBannoe2019, Bannoe2021}.
During the work on the dissertation, the author also published 2 articles \cite{LebedevDolininaHong, ZezyulinLebedevAlfimovMalomed} on related topics (results of one of them were also reported on conference \cite{LebedevAlfimovBannoe2019}).

\textbf{Personal contribution of the author.}
The main finding of the thesis was obtained either by the applicant in person, or in collaboration with co-authors where the role of the applicant was dominant.
The numerical implementations of all the algorithms and other computer programs was fulfilled by the applicant personally.

\textbf{Structure and volume of the dissertation.}
The thesis consists of introduction, four chapters, conclusion, three appendices, and a bibliography.
Total volume of the dissertation is 130 pages.
Among them there are 115 pages of text, including 35 figures, 4 tables and 1 algorithm scheme.
Bibliography consists of 61 titles.

\textbf{\hyperref[introduction]{Introduction}} contains formulation of the problem, justification of the relevance of the research topic, tasks and objectives of the study, scientific novelty, and the highlights of the thesis.

In \textbf{\hyperref[chapter:I]{Chapter I}} several propositions about regular and singular solutions of equation \eqref{eq:intro-stationary} are formulated and proved.

\textbf{\hyperref[chapter:II]{Chapter II}} contains the theory lying behind the symbolic dynamics approach that is used to classify solutions of equation \eqref{eq:intro-stationary}.
Main theorem of the approach is formulated and proved.
Numerical procedure that establishes the possibility of application of this approach is provided.

In \textbf{\hyperref[chapter:III]{Chapter III}} the set of stationary localized solutions has been studied for the case of equation \eqref{eq:intro-gpe} when no trapping potential is included, $U(x) \equiv 0$, and the pseudopotential has a cosine form, $P(x) = A + \cos 2x$.
Linear stability of these solutions is analysed by means of spectral method which is also briefly described.

In \textbf{\hyperref[chapter:IV]{Chapter IV}} the set of SLMs is studied for equation \eqref{eq:intro-gpe} where, along with the periodic pseudopotential, the trapping potential in the form of harmonic potential well is included.
Stability of low-amplitude solutions is studied by means of asymptotic methods.

In \textbf{\hyperref[conclusion]{Conclusion}} the results of the thesis are summarized.

In \textbf{\hyperref[appendix:lemma-on-bounded-solutions]{Appendix A}} Lemma on bounded solutions is proved.
This lemma is used in Chapter I.

\textbf{\hyperref[appendix:solutions-of-duffing-equations]{Appendix B}} contains explicit solutions for two equations of the Duffing oscillator type: 
\begin{equation}
	u_{xx} - u + u^3 = 0; \quad u_{xx} - u - u^3 = 0,
\end{equation}
that are used in Section~\ref{section:poincare-map-domains-piecewise}.

In \textbf{\hyperref[appendix:strips-mapping-theorems]{Appendix C}} Theorems on h- and v-strips mapping are proved.
These theorems are used in Chapter~\ref{chapter:II}.
