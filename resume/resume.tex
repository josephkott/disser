% TODO: Replace $A$, $B$ with $\alpha$, $\beta$?
% TODO: Add several sentences about the "coding set".
% TDOO: Add several sentences about two scales.
\documentclass[candidate, href, colorlinks]{disser}

\usepackage [T2A] {fontenc}
\usepackage [utf8] {inputenc}
\usepackage [english, main = russian] {babel}
\usepackage {tabularx}

\usepackage [intlimits] {amsmath}
\usepackage {amssymb}
\usepackage {amsfonts}
\usepackage [autostyle] {csquotes}
\usepackage {xcolor}

\usepackage[a4paper, mag=1000, left=2.5cm, right=1cm, top=2cm, bottom=2cm, headsep=0.7cm, footskip=1cm] {geometry}

\usepackage[style=gost-numeric, backend=biber, language=auto, hyperref=auto, autolang=other, defernumbers, sorting=none]{biblatex}

\addbibresource{bibliography.bib}

\pagestyle{footcenter} % page number at bottom center
% \chapterpagestyle{footcenter} % (?)

\renewcommand*{\thefootnote}{[\arabic{footnote}]}

\begin{document}

% -------------------- TITLE -------------------- %
\begin{titlepage}
\thispagestyle{empty}
\enlargethispage{1cm}
\vspace*{-2cm}

\begin{center}
	Институт математики с вычислительным центром \\ УФИЦ РАН
\end{center}

\vskip1cm
	
\begin{flushright}
	\emph{На правах рукописи}
\end{flushright}
	
\vskip3cm

\begin{center}
	{\large Лебедев Михаил Евгеньевич}
	\vskip1cm
	{\Large\bfseries Стационарные решения уравнения Гросса--Питаевского с периодически модулированной нелинейностью \par}
	\vskip1.5cm
	{РЕЗЮМЕ ДИССЕРТАЦИИ \\ на соискание ученой степени кандидата наук \\ по прикладной математике}
\end{center}

\vskip2cm

\hspace{8cm}\begin{minipage}{0.4\linewidth}
	Научный руководитель \\
	д.~ф.-м.~н., проф. \\
	Алфимов Георгий Леонидович
\end{minipage}

\vfill

\begin{center}
	{Москва -- \the\year}
\end{center}

\normalfont\clearpage
\end{titlepage}
% -------------------- END -------------------- %

\thispagestyle{empty}
\vspace*{-2cm}
\noindent
\begin{center}
Работа выполнена в \emph{ИМВЦ УФИЦ РАН}.
\end{center}
\vskip1ex\noindent
\begin{tabularx}{\linewidth}{@{}lX@{}}
	\textbf{Научный руководитель:} & \textbf{Алфимов Георгий Леонидович}\\
	& доктор физико-математических наук \\[6pt]
	
	\textbf{Официальные оппоненты:} & \textbf{\textcolor{red}{фамилия имя отчество}}\\
	& \textcolor{red}{ученая степень, ученое звание} \\[6pt]
	& \textbf{\textcolor{red}{фамилия имя отчество}} \\
	& \textcolor{red}{ученая степень, ученое звание} \\[6pt]
	
	\textbf{Ведущая организация:} & Институт математики с вычислительным центром УФИЦ РАН
\end{tabularx}

\vskip5ex\noindent
Защита состоится \datefield{} в \rule[0pt]{1cm}{0.5pt} часов
на заседании диссертационного совета \emph{\textcolor{red}{шифр совета}} при \emph{\textcolor{red}{название организации, при которой создан совет}}, расположенном по адресу: \emph{\textcolor{red}{адрес}}.

\vskip1ex\noindent
С диссертацией можно ознакомиться в библиотеке \emph{\textcolor{red}{название организации}}, а также по ссылке \href{http://www.xyz.com}{\textcolor{red}{http://www.xyz.com}}.

\vskip1ex\noindent
Автореферат разослан \datefield{}

\vskip2ex\noindent
Отзывы и замечания по автореферату в двух экземплярах, заверенные
печатью, просьба высылать по вышеуказанному адресу на имя ученого секретаря диссертационного совета.

\vfill\noindent
\begin{minipage}[b]{1\linewidth}
	Ученый секретарь \\
	диссертационного совета, \\
	\emph{кандидат физико-математических наук}, \emph{доцент} \hfill \emph{Самбурский Л. М.}
\end{minipage}

\clearpage

\nsection{Общая характеристика работы}

\textbf{Введение.}
Начиная с 90-х годов прошлого века, нелинейное уравнение Шрёдингера (НУШ) с дополнительной пространственной неавтономностью продолжает оставаться объектом пристального изучения.
Интерес к этому классу уравнений во многом обусловлен экспериментальными успехами в исследовании конденсата Бозе--Эйнштейна (БЭК).
Явление конденсата вещества при сверхнизких температурах было предсказано в работах Бозе и Эйнштейна в 1924 году\footnote{A. Einstein, ``Quantentheorie des einatomigen idealen Gases'', Preussische Akademie der Wissenschaften, Berlin, 1924.}.
Экспериментально такое состояние вещества получено впервые в 1995 году независимо двумя группами исследователей\footnote{M. H. Anderson, J. R. Ensher, M. R. Matthews, C. E. Wieman, E. A. Cornell, ``Observation of Bose--Einstein Condensation in a Dilute Atomic Vapor'', Science, (New Series), Vol. 269, P. 198-201, 1995.}.
В 2001 году это открытие было удостоено Нобелевской Премии.

Возможность получение БЭК стимулировала экспериментальные и теоретические исследования по всему миру, открывшие целый ряд потенциальных практических приложений.
В частности, ожидается, что использование БЭК должно привести к появлению новых высокочастотных интерферометров\footnote{C. Gross, T. Zibold, E. Nicklas, J. Esteve, and M. K. Oberthaler, ``Nonlinear atom interferometer surpasses classical precision limit'', Nature, Vol. 464, P. 1165--1169, 2010.}.
Также БЭК может быть использован для построения квантовых компьютеров\footnote{D. Jaksch, P. Zoller, ``The cold atom Hubbard toolbox'', Annals of Physics, Vol. 315, P. 52--79, 2005.} и квантовых лазеров\footnote{W. Guerin, J.-F. Riou, J. P. Gaebler, V. Josse, P. Bouyer, and A. Aspect, ``Guided Quasicontinuous Atom Laser'', Phys. Rev. Lett., Vol. 97, P. 200402, 2006.}.

Динамика БЭК в приближении среднего поля описывается уравнением шредингеровского типа с пространственной неавтономностью.
\begin{equation}
	i \Psi_t + \Psi_{xx} - U(x) \Psi + P(x) |\Psi^2| \Psi = 0.
\label{eq:gpe}
\end{equation}
В контексте теории БЭК уравнение типа \eqref{eq:gpe} носит название {\it уравнения Гросса -- Питаевского}.
Здесь $\Psi(t, x)$ представляет обезразмеренную волновую функцию конденсата, функция $U(x)$ описывает потенциал ловушки, удерживающей конденсат, а $P(x)$ соответствует нелинейному потенциалу (или {\it псевдопотенциалу}).
Псевдопотенциал $P(x)$ описывает зависимость длины рассеяния частиц конденсата от пространственной координаты, которая может быть переменной величиной, что достигается различными техниками, как например, использованием так называемого резонанса Фешбаха\footnote{C. Chin, R. Grimm, P. Julienne, and E. Tiesinga, ``Feshbach resonances in ultracold gases'', Rev. Mod. Phys. Vol. 82, P. 1225, 2010.}.
Интервалы с положительным значением псевдопотенциала $P(x) > 0$ соответствуют случаю межатомного притяжения, в то время как интервалы с отрицательным значением, $P(x) < 0$, ---  межатомному отталкиванию частиц конденсата.
Классическими модельными примерами потенциала $U(x)$ является гармонический потенциал $U(x) = Ax^2$ (магнитная ловушка) и периодический потенциал $U(x) = A \cos 2x$ (оптическая ловушка).
В качестве модельных примеров псевдопотенциала $P(x)$ используются различные периодические функции (нелинейная решетка), как, например, косинусный потенциал $P(x) = A + B \cos \Omega x$\footnote{H. Sakaguchi,  B. A. Malomed, ``Matter-wave solitons in nonlinear optical lat­tices'', Phys. Rev. E, Vol. 72, P. 046610, 2005.}.

Для физических приложений важную роль играют решения уравнения \eqref{eq:gpe} специального вида --- так называемые {\it стационарные локализованные решения} ({\it стационарные локализованные моды}, СЛМ).
Они получаются в результате подстановки в соответствующее уравнение \eqref{eq:gpe} выражения
\begin{equation}
	\Psi(t, x) = u(x) e^{-i \omega t},
\label{eq:ansatz}
\end{equation}
где функция $u(x)$ удовлетворяет условию локализации, $u(x) \to 0$ при $x \to \pm \infty$, а $\omega$ есть вещественный параметр, имеющий смысл химического потенциала конденсата.
Профиль стационарного локализованного решения $u(x)$, действительнозначная функция\footnote{G. L. Alfimov, V. V. Konotop, and M. Salerno, ``Matter solitons in Bose--Einstein condensates with optical lattices'', Europhys. Lett., Vol. 58, P. 7--13, 2002}, удовлетворяет уравнению
\begin{equation}
	u_{xx} + (\omega - U(x)) u + P(x) u^3 = 0.
\label{eq:stationary}
\end{equation}

Стоит отметить, что далеко не все локализованные решения уравнения \eqref{eq:stationary} одинаково интересны с физической точки зрения.
В частности, особо важным свойством является устойчивость локализованных решений.
Если СЛМ является неустойчивой, малые возмущения такого решения приводят к его разрушению при эволюции, описываемой уравнением \eqref{eq:gpe}.
Поэтому именно устойчивые локализованные решения особенно ценны для различных физических приложений, а сама проверка СЛМ на устойчивость является существенной частью их теоретического исследования.

\textbf{Постановка проблемы.}
Итак, при изучении динамики, описываемой уравнением \eqref{eq:gpe}, естественным образом возникают следующие вопросы:
\begin{enumerate}
	\item Возможно ли перечислить {\it полностью все} стационарные локализованные решения уравнения \eqref{eq:gpe}, одновременно существующих при заданных параметрах уравнения?
	\item Какие из перечисленных решений являются устойчивыми?
\end{enumerate}

\textbf{Степень разработанности темы исследования.}
Стоит отметить, что в большинстве работ, посвящённых данной тематике вопрос о поиске / описании {\it всех} СЛМ не ставится, и вместо него рассматривается вопрос об отдельных классах СЛМ, соответствующих той или иной физической структуре, см. обзор\footnote{Y. V. Kartashov, B. A. Malomed, and L. Torner, ``Solitons in nonlinear lattices'', Rev. Mod. Phys. Vol. 83, P. 247, 2011.}.
В то же время, несмотря на некоторую <<амбициозность>> вопроса (1), сочетание строгих аналитических утверждений с численным счётом позволяет добиться существенных результатов в этом направлении. 
Отметим некоторое количество важных результатов.

Для уравнения \eqref{eq:stationary} c потенциалом $U(x)$, имеющего вид бесконечной потенциальной ямы, $U(x) = A x^2$, для случая отталкивающий взаимодействий, $P(x) \equiv -1$, был предложен метод <<доказательных вычислений>>, позволяющий гарантировать нахождение {\it всех} ограниченных решений\footnote{G. L. Alfimov, D. A. Zezyulin, ``Nonlinear modes for the Gross--Pitaevskii equation --- a demonstrative computational approach'', Nonlinearity, Vol. 20, P. 2075--2092, 2007.}.
Разработанный метод впоследствии был обобщен на системы из нескольких связанных уравнений Гросса--Питаевского, в которых соответствующие псевдопотенциалу коэффициенты также не зависят от пространственной координаты\footnote{G. L. Alfimov, I. V. Barashenkov, A. P. Fedotov, V. V. Smirnov, D. A. Zezyulin, ``Global search for localised modes in scalar and vector nonlinear Schr{\"o}dinger-type equations'', Physica D, Vol. 397, P. 39--53, 2019.}.

Для периодического потенциала $U(x)$, как, например, $U(x) = A \cos 2x$, в случае исключительно отталкивающих взаимодействий частиц конденсата $P(x) \equiv -1$ предложены достаточные условия, опять же допускающие исчерпывающее описание {\it всех} ограниченных решений уравнения \eqref{eq:stationary}.
При этом показано, что выполнения этих условий позволяет установить взаимно-однозначное соответствие между ограниченными решениями и всевозможными бесконечными в обе стороны последовательностями символов из некоторого конечного алфавита\footnote{\label{note:alfavr} G. L. Alfimov, A. I. Avramenko, ``Coding of nonlinear states for the Gross--Pitaevskii equation with periodic potential'', Physica D, Vol. 254, P. 29--45, 2013.}.
Последовательности такого рода названы авторами работы {\it кодами решений}, а сам процесс присвоения кодов -- {\it кодированием решений}.
Проверка достаточных условий проводилась авторами работы с помощью численного счёта.
Опираясь на результаты предыдущей работы в другой работе\footnote{G. L. Alfimov, P. P. Kizin, D. A. Zezyulin, ``Gap solitons for the repulsive Gross-Pitaevskii equation with periodic potential: Coding and method for computation'', Discrete and Continuous Dynamical Systems --- Series B, Vol. 22, P. 1207--1229, 2017.} был разработан алгоритм, позволяющий по коду решения численно построить его профиль.

\textbf{Актуальность темы исследования.}
Актуальной задачей является обобщение результатов приведенных выше работ на случай переменного псевдопотенциала $P(x) \neq \mathrm{const}$.
В частности, перспективным направлением исследования является обобщение аппарата кодирования решений на случай периодических потенциала и псевдопотенциала.
Детальная классификация решений уравнения \eqref{eq:gpe} открывает возможность обнаружения новых, ранее неизвестных устойчивых СЛМ.

\textbf{Цели и задачи диссертационной работы.}
Основным объектом исследования данной диссертационной работы являются стационарные решения одномерного уравнения Гросса--Питаевского \eqref{eq:gpe} с {\it периодическим псевдопотенциалом}.
Цели и задачи работы можно сформулировать следующим образом:
\begin{enumerate}
	\item Обобщить подход, связанный с кодированием решений\textsuperscript{\ref{note:alfavr}}, на случай периодического потенциала $U(x)$ и периодического псевдопотенциала $P(x)$.
		Сформулировать достаточные условия, дающие возможность применить такой подход, а также указать способ проверки этих условий (аналитически или с помощью численного счета).
	\item Исследовать множество стационарных решений в случае принципиально нелинейных взаимодействий, когда линейным потенциалом можно пренебречь, $U(x) \equiv 0$, 
	\item Для случая бесконечной потенциальной ямы $U(x) = A x^2$ исследовать влияние периодического псевдопотенциала на структуру множества стационарных локализованных решений и их устойчивость.
\end{enumerate}

\textbf{Методология и методы исследования.}
Для исследования возможных типов СЛМ в работе используется так называемый <<метод исключения сингулярных решений>>\textsuperscript{\ref{note:alfavr}}.
Решение уравнения \eqref{eq:stationary} называется {\it сингулярным}, если оно уходит на бесконечность в конечной точке числовой прямой:
\begin{equation}
	\lim \limits_{x \to x_0} u(x) = \infty.
\end{equation}
При выполнении определенных условий <<большая часть>> решений уравнения \eqref{eq:stationary} представляет собой сингулярные решения.
Множество оставшихся решений, называемых {\it регулярными}, оказывается достаточно <<бедным>> и может быть полностью описано в терминах символической динамики.

Решение дифференциального уравнения \eqref{eq:stationary} производится с помощью метода Рунге--Кутта четвертого порядка точности.
Для построения локализованных решений уравнения \eqref{eq:stationary} в работе используется метод стрельбы.
Устойчивость построенных решений проверяется с помощью метода коллокаций Фурье\footnote{J. Yang, ``Nonlinear Waves in Integrable and Nonintegrable Systems'', Philadelphia: SIAM, 2010.}, а также посредством эволюционного моделирования динамики уравнения \eqref{eq:gpe} с помощью консервативной конечно-разностной схемы\footnote{V. Trofimov, N. Peskov Comparison of finite-difference schemes for the Gross-Pi­ taevskii equation // Mathematical Modelling and Analysis. — 2009. — Mar. — Vol. 14. — P. 109–126.}.
Все алгоритмы и численные методы реализованы в среде {\tt MATLAB} с использованием расширения {\tt MEX} для поддержки высокопроизводительных вычислений.

\textbf{Научная новизна.}
Доказан ряд общих утверждение, указывающих, когда уравнение \eqref{eq:stationary} допускает существование сингулярных решений, а также, когда все его решения регулярны.
В частности, показано, что если псевдопотенциал принимает отрицательное значение хотя бы в одной точке $x_0$, $P(x_0) < 0$, то существуют два однопараметрических семейства решений, уходящих на бесконечность в этой точке, т.е. $\lim \limits_{x \to x_0} u(x) = \infty$, а также получены асимптотические формулы для этих семейств.

Метод исключения сингулярный решений получает дальнейшее развитие.
В работе предложены достаточные условия существования взаимно-однозначного соответствия между регулярными решениями уравнения \eqref{eq:stationary} и бесконечными последовательностями символов над некоторым алфавитом.
В отличии от ранее полученных результатов\textsuperscript{\ref{note:alfavr}}, предложенные достаточные условия могут быть эффективно проверены с помощью численного счета.
В диссертации приводится алгоритм их численной проверки, а также его теоретическое обоснование.

Для случая $U(x) \equiv 0$ и модельного периодического псевдопотенциала вида $P(x) = A + B \cos 2x$ было исследовано множество стационарных локализованных решений.
Использование выше упомянутых техник позволило эффективно описать множество СЛМ и в конечном счёте обнаружить новое устойчивое решение, которое ранее не обсуждалось в литературе при рассмотрении задач, связанных с уравнением \eqref{eq:gpe}.
Найденное новое устойчивое решение получило называние {\it дипольный солитон} \cite{LebedevAlfimovMalomed}.

Наконец, в данной работе изучен вопрос о влиянии периодического псевдопотенциала вида $P(x) = A + B \cos \Omega x$ на множество СЛМ в случае бесконечной потенциальной ямы, $U(x) = A x^2$.
Показано, что по сравнению с хорошо изученным случаем $P(x) = \mathrm{const}$, множество стационарных локализованных решений оказывается значительно богаче, а именно, появляются существенно нелинейные решения, которые не существуют в малоамплитудном пределе.
Исследована зависимость устойчивости СЛМ от частоты псевдопотенциала $\Omega$.
Для псевдопотенциала с нулевой средней, $P(x) = B \cos \Omega x$, показано, что увеличение частоты позволяет стабилизировать малоамплитудные решения, чьи аналоги в модели с $P(x) = \mathrm{const}$ оказываются неустойчивыми.

% \textbf{Теоретическая и практическая значимость.}
% Результаты, изложенные в диссертации, могут быть использованы при проведении экспериментов с конденсатом Бозе--Эйнштейна.

\textbf{Положения, выносимые на защиту:}
\begin{enumerate}
	\item Сформулированы и доказаны общие утверждения о наличии и отсутствии сингулярных решений уравнения \eqref{eq:stationary}.
		Показано, что в случае $P(x) > 0$ все решения уравнения \eqref{eq:stationary} регулярны.
		Если $P(x)$ принимает отрицательное значение хотя бы в одной точке $x_0$, $P(x_0) < 0$, то существуют два однопараметрических семейства решений, уходящих на бесконечность в точке $x = x_0$; построена асимптотика этих решений. 
		В том случае, когда $Q(x) < 0$ и $P(x) < 0$, показано, что все решения уравнения \eqref{eq:stationary} сингулярны.
	\item Сформулированы достаточные условия возможности кодирования регулярных решений уравнения \eqref{eq:stationary} и предложен эффективный алгоритм численной проверки этих условий.
	\item Для случая $U(x) \equiv 0$, $P(x) = A + B \cos 2x$ исследовано множество СЛМ и обнаружено новое устойчивое локализованное решение уравнения \eqref{eq:gpe} --- {\it дипольный солитон}.
	\item В случае бесконечной потенциально ямы показано, что присутствие периодического псевдопотенциала приводит к появлению новых классов СЛМ, не имеющих аналогов в моделях с $P(x) = \mathrm{const}$.
		Для псевдопотенциала с нулевой средней установлено, что частота псевдопотенциала существенным образом влияет на устойчивость СЛМ, а именно, увеличение частоты приводит к стабилизации малоамплитудных решений.
\end{enumerate}

\textbf{Степень достоверности и апробация результатов.}
Модель Гросса -- Питаевского является классической задачей из физики сверхнизких температур и её достоверность не вызывает сомнений.
В данной диссертационной работе численно строятся локализованные стационарные решения указанной модели, а также численно исследуется устойчивость таких решений.
Построение решений производится при помощи стандартных численных методов с контролируемой точностью.
Исследование устойчивости построенных решений производится методом коллокаций Фурье, который хорошо зарекомендовал себя для решения подобных задач.
Результаты исследования устойчивости проверяются численным решением эволюционной задачи при помощи конечно-разностной схемы.
Основные результаты диссертационной работы докладывались на различных научных семинарах и конференциях, в числе которых:
\begin{enumerate}
	\item <<Гамильтонова динамика, неавтономные системы и структуры в уравнениях с частными производными>>, Нижегородский государственный университет, Нижний Новгород, декабрь, 2014 г.
	\item <<Фундаментальная математика и ее приложения в естествознании>>, БашГУ, Уфа, сентябрь, 2015 г.
	\item <<Динамика, бифуркации и странные аттракторы>>, Нижегородский государственный университет, Нижний Новгород, июль, 2016 г.
	\item <<Комплексный анализ, математическая физика и нелинейные уравнения>>, Башкортостан, оз. Банное, март, 2018 г.
	\item ``Nonlinear Phenomena in Bose Condensates and Optical Systems'', Ташкент, Узбекистан, август, 2018 г.
	\item <<Комплексный анализ, математическая физика и нелинейные уравнения>>, Башкортостан, оз. Банное, март, 2019 г.
	\item <<Комплексный анализ, математическая физика и нелинейные уравнения>>, Башкортостан, оз. Банное, март, 2021 г.
\end{enumerate}

\textbf{Публикации.}
Материалы диссертации опубликованы в 10 печатных работах, из них 3 статьи в рецензируемых журналах \cite{AlfimovLebedev, LebedevAlfimovMalomed, AlfimovGegelLebedevMalomedZezyulin} и 7 тезисов докладов на различных конференциях \cite{NizhniNovgorod2014, Ufa2015, NizhniNovgorod2016, Bannoe2018, Tashkent2018, Bannoe2019, Bannoe2021}.
Также за время работы над диссертацией автором было опубликовано 2 статьи\footnote{M. E. Lebedev, D. A. Dolinina, K. B. Hong, {\it et al.}, ``Exciton-polariton Josephson junctions at finite temperatures'', Scientific Reports Vol. 7, P. 9515, 2017}\textsuperscript{,}\footnote{D. A. Zezyulin, M. E. Lebedev, G. L. Alfimov, and B. A. Malomed, ``Symmetry breaking in competing single-well linear-nonlinear potentials'', Phys. Rev. E, Vol. 98, P. 042209, 2018.} и 1 тезис доклада\footnote{D. A. Zezyulin, M. E. Lebedev, G. L. Alfimov, and B. A. Malomed, ``Symmetry breaking in competing single-well linear-nonlinear potentials'', Тезисы доклада на конференции <<Комплексный анализ, математическая физика и нелинейные уравнения>>, Башкортостан, оз. Банное, март, 2019.} по смежным тематикам.

\textbf{Личный вклад автора.}
Содержание диссертации и основные положения, выносимые на защиту, отражают персональный вклад автора в опубликованные работы.
Подготовка к публикации полученных результатов проводилась совместно с соавторами, причем вклад диссертанта был определяющим.
Все представленные в диссертации результаты получены лично автором с использованием разработанных методов и компьютерных программ.

\textbf{Структура и объем диссертации.}
Диссертация состоит из введения, четырёх глав, трёх приложений и библиографии.
Общий объем диссертации \textcolor{red}{?} страниц, из них \textcolor{red}{?} страниц текста, включая \textcolor{red}{?} рисунков, четыре таблицы и одну схему алгоритма.
Библиография включает \textcolor{red}{?} наименования на \textcolor{red}{?} страницах.

\nsection{Содержание работы}

\textbf{Во введении (Introduction)} обоснована актуальность диссертационной работы, сформулирована цель и аргументирована научная новизна исследований, показана практическая значимость полученных результатов, представлены выносимые на защиту научные положения.

\textbf{В первой главе (Chapter I)}

Результаты первой главы опубликованы в работах \cite{AlfimovLebedev}, \cite{NizhniNovgorod2014} и \cite{Ufa2015}.

\textbf{Во второй главе (Chapter II)}

Результаты второй главы опубликованы в работах \cite{Bannoe2019} и \cite{Bannoe2021}.

\textbf{В третьей главе (Chapter III)}

Результаты третьей главы опубликованы в работах \cite{LebedevAlfimovMalomed}, \cite{NizhniNovgorod2016} и \cite{Tashkent2018}.

\textbf{В четвертой главе (Chapter IV)}

Результаты четвертой главы опубликованы в работах \cite{AlfimovGegelLebedevMalomedZezyulin} и \cite{Bannoe2018}.

\textbf{В заключении (Сonclusion)}

\textbf{В приложении A (Appendix A)}

\textbf{В приложении B (Appendix B)}

\textbf{В приложении C (Appendix C)}

\renewcommand{\bibname}{\protect\leftline{Bibliography}}

\renewcommand{\bibname}{\protect\leftline{\large Список публикаций автора по теме диссертации}}
\printbibliography[keyword=own]

% Center aligned bibliography title
% \printbibliography[keyword=own, title={\large Список публикаций автора по теме диссертации}]

% All other cited literature
% \printbibliography[notkeyword=own, title={Цитированная литература}]

\end{document}
