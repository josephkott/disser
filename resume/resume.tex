\documentclass[candidate, href, colorlinks]{disser}

\usepackage [T2A] {fontenc}
\usepackage [utf8] {inputenc}
\usepackage [english, main = russian] {babel}
\usepackage {tabularx}

\usepackage [intlimits] {amsmath}
\usepackage {amssymb}
\usepackage {amsfonts}
\usepackage [autostyle] {csquotes}
\usepackage {xcolor}

\usepackage[a4paper, mag=1000, left=2.5cm, right=1cm, top=2cm, bottom=2cm, headsep=0.7cm, footskip=1cm] {geometry}

\usepackage[style=gost-numeric, backend=biber, language=auto, hyperref=auto, autolang=other, defernumbers, sorting=none]{biblatex}

\addbibresource{bibliography.bib}

\pagestyle{footcenter} % page number at bottom center
%\chapterpagestyle{footcenter} % (?)

\begin{document}

% -------------------- TITLE -------------------- %
\begin{titlepage}
\thispagestyle{empty}
\enlargethispage{1cm}
\vspace*{-2cm}

\begin{center}
	Институт математики с вычислительным центром \\ УФИЦ РАН
\end{center}

\vskip1cm
	
\begin{flushright}
	\emph{На правах рукописи}
\end{flushright}
	
\vskip3cm

\begin{center}
	{\large Лебедев Михаил Евгеньевич}
	\vskip1cm
	{\Large\bfseries Стационарные решения уравнения Гросса--Питаевского с периодически модулированной нелинейностью \par}
	\vskip1.5cm
	{РЕЗЮМЕ ДИССЕРТАЦИИ \\ на соискание ученой степени кандидата наук \\ по прикладной математике}
\end{center}

\vskip2cm

\hspace{8cm}\begin{minipage}{0.4\linewidth}
	Научный руководитель \\
	д.~ф.-м.~н., проф. \\
	Алфимов Георгий Леонидович
\end{minipage}

\vfill

\begin{center}
	{Москва -- \the\year}
\end{center}

\normalfont\clearpage
\end{titlepage}
% -------------------- END -------------------- %

\thispagestyle{empty}
\vspace*{-2cm}
\noindent
\begin{center}
Работа выполнена в \emph{Институте математики с вычислительным центром УФИЦ РАН}.
\end{center}
\vskip1ex\noindent
\begin{tabularx}{\linewidth}{@{}lX@{}}
	\textbf{Научный руководитель:} & \textbf{Алфимов Георгий Леонидович}\\
	& доктор физико-математических наук \\[6pt]
	
	\textbf{Официальные оппоненты:} & \textbf{\textcolor{red}{фамилия имя отчество}}\\
	& \textcolor{red}{ученая степень, ученое звание} \\[6pt]
	& \textbf{\textcolor{red}{фамилия имя отчество}} \\
	& \textcolor{red}{ученая степень, ученое звание} \\[6pt]
	& \textbf{\textcolor{red}{фамилия имя отчество}} \\
	& \textcolor{red}{ученая степень, ученое звание} \\[6pt]
	
	\textbf{Ведущая организация:} & Институт математики с вычислительным центром УФИЦ РАН
\end{tabularx}

\vskip5ex\noindent
Защита состоится \datefield{} в \rule[0pt]{1cm}{0.5pt} часов
на заседании диссертационного совета \emph{\textcolor{red}{шифр совета}} при \emph{\textcolor{red}{название организации, при которой создан совет}}, расположенном по адресу:\emph{\textcolor{red}{адрес}}

\vskip1ex\noindent
С диссертацией можно ознакомиться в библиотеке \emph{\textcolor{red}{название организации}}, а также по ссылке \href{http://www.xyz.com}{\textcolor{red}{http://www.xyz.com}}.

\vskip1ex\noindent
Автореферат разослан \datefield{}

\vskip2ex\noindent
Отзывы и замечания по автореферату в двух экземплярах, заверенные
печатью, просьба высылать по вышеуказанному адресу на имя ученого секретаря диссертационного совета.

\vfill\noindent
\begin{minipage}[b]{1\linewidth}
	Ученый секретарь \\
	диссертационного совета, \\
	\emph{доктор физико-математических наук}, \emph{профессор} \hfill \emph{\textcolor{red}{Фамилия И. О.}}
\end{minipage}

%\vfill\noindent
%\begin{minipage}[b]{0.4\linewidth}
%  Ученый секретарь \\
%  диссертационного совета, \\
%  \emph{ученая степень}, \emph{ученое звание}
% %\hfill \emph{фамилия и. о.}
%\end{minipage}
%\begin{minipage}[b]{0.4\linewidth}
%  \hfill \\
%  \hfill \\
%  \hfill asdasd
%\end{minipage}
%\hfill

%% Insert a file with the facsimile of scientific secretary
%\makeatletter
%\ifDis@facsimile
%  \includegraphics[width=3cm]{sec-facsimile}\hfill
%\fi%
%\makeatother%
%\emph{фамилия и. о.}

\clearpage

\nsection{Общая характеристика работы}

\textbf{Актуальность темы исследования.} % (?)

\textbf{Степень разработанности темы исследования.}{

\textbf{Цели и задачи диссертационной работы:}

\textbf{Научная новизна.}

\textbf{Теоретическая и практическая значимость.}

\textbf{Методология и методы исследования.} % (?)

\textbf{Положения, выносимые на защиту:}

\textbf{Степень достоверности и апробация результатов.}

\textbf{Публикации.}

\textbf{Личный вклад автора.}

\textbf{Структура и объем диссертации.}

\nsection{Содержание работы}

\textbf{Во Введении} обоснована актуальность диссертационной работы, сформулирована цель и аргументирована научная новизна исследований, показана практическая значимость полученных результатов, представлены выносимые на защиту научные положения.

\textbf{В первой главе}

Результаты первой главы опубликованы в работе [...].

\textbf{Во второй главе}

Результаты второй главы опубликованы в работе [...].

\textbf{В третьей главе}

Результаты третьей главы опубликованы в работе [...].

\textbf{В четвертой главе}

Результаты четвертой главы опубликованы в работе [...].

\textbf{В Заключении}

\printbibliography[keyword=own, title={Основные публикации по теме диссертации}]

% All other cited literature
% \printbibliography[notkeyword=own, title={Цитированная литература}]

\end{document}
