\chapter*{Conclusion}
\addcontentsline{toc}{chapter}{Conclusion}
\label{conclusion}

In the present work one-dimensional stationary states of Bose--Einstein condensate cloud were studied.
Dynamics of the condensate in so-called mean-field ap­proximation is well described by the Schr\"odinger type equation \eqref{eq:intro-gpe} with spatial non-au­tonomous terms.
One of this term, $U(x)$, corresponds to the potential that is used to confine condensate.
Another term, $P(x)$, called pseudopotential, describes the dependence of scattering length on the coordinates that can be made non-constant by various experimental techniques, such as Feshbach resonance.

In our study we focused on special subclass of the solutions, so-called stationary localized modes, SLMs, satisfying the ansatz \eqref{eq:intro-ansatz} and the localization conditions \eqref{eq:intro-localization}.
Profile of the stationary solutions satisfy equation \eqref{eq:intro-stationary}.
Our goal was to identify stationary solutions for equation \eqref{eq:intro-gpe} and study their stability.
To describe stationary solutions for the case of both periodic potential and pseudopotential functions we developed a symbolic dynamics framework.
It allows under certain conditions to classify all solutions of equation \eqref{eq:intro-stationary} with bi-infinite symbolic sequences, so-called codes.
Based on the structure of coding sets $\mathscr{U}_L^{\pm}$, which are the Poicar\'e map domains for equation \eqref{eq:intro-stationary}, one can assign a symbolic code to each stationary solution.
Besides, under specified conditions such correspondence is bijective.
This approach was applied to the cases of piecewise and cosine-form pseudopotential.
It turns out that localized solutions family for such model is quite rich.
For cosine-form pseudopotential our comprehensive study reveled new types of stable solutions.

For the case of harmonic potential trap and periodic pseudopotential family of SLMs was studied.
Since harmonic potential is not periodic, coding approach cannot be applied to such problem.
Numerical analysis showed that presence of periodic pseudopotential significantly enrich the family of localized solutions.
We also studied stability of low-amplitude localized solutions with asymptotic methods and describe the impact of the pseudopotential frequency on their  stability.
Stability of found stationary localized solution was studied numerically with spectral method and confirmed by numerical simulation with finite-difference scheme.

Let's summarize in details main results of the thesis.
\begin{enumerate}
	\item The statements about the presence and absence of singular solutions of equation \eqref{eq:intro-stationary} are formulated and proved.
		It is shown that in the case $P(x) > 0$ all solutions of \eqref{eq:intro-stationary} are regular.
		If $P(x)$ is negative at the point $x_0$, $P(x_0) < 0$, then there exist two one-parametric families of solutions, which tend to infinity at the point $x_0$.
		In the case $Q(x) < 0$ and $P(x) < 0$, it is shown that all solutions of equation \eqref{eq:intro-stationary} are singular.
	\item We developed an approach of one-to-one coding of solutions for equation \eqref{eq:intro-stationary} with periodic functions $Q(x)$, $P(x)$.
		This approach requires some condition to be met.
		We formulated these condition in a form of two hypothesis and provided an efficient algorithm for their numerical verification.
	\item For the case $U(x) \equiv 0$, $P(x) = A + \cos 2x$ the set of stationary localized modes for equation \eqref{eq:intro-gpe} has been studied.
		A new stable previously unknown solution, called {\it dipole soliton}, was found.
	\item In the case of harmonic potential well it was shown that including of periodic pseudopotential results in new classes of SLMs without linear counterpart.
		For the pseudopotential with zero mean, it was concluded that the increase of pseudopotential frequency stabilize low-amplitude solutions.
\end{enumerate}

In conclusion, let's also consider possible directions in which the current work can be further developed and generalized.
One such direction is to consider a model described by GPE \eqref{eq:intro-gpe} for both periodic potential and pseudopotential with different spatial frequencies.
Our coding approach can be applied to such case.
It's interesting to study family of stationary localized solutions and their stability in a view of interplay between these two spatial frequencies.

Another generalization of this study is related to two component BEC models \cite{TrippenbachGoralRzazewskiMalomedBand, OhbergSantos, SvidzinskyChui, NavarroCarreteroGonzalezKevrekidis, KevrekidisFrantzeskakis2016}.
These models describe mixtures of ultracold atoms of two different types.
It would be interesting to extend the methods developed in this work by considering four-dimensional plane of initial data and analysing counterparts of coding sets in this case. 
This study may result in classification of vector solitons admitted by the system under given parameters.
The next step of this perspective research would be the analysis of their stability.
It may be promising, since it was shown that including nonlinear lattice in one component of the two component model may stabilize two-dimensional vector solitons \cite{BorovkovaMalomedKartashov}.
Following the setup of the last paper, it would be quite interesting to extend the approach developed in this work to multidimensional problems.
