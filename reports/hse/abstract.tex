\documentclass [a4paper] {article}

\usepackage [T2A] {fontenc}
\usepackage [utf8] {inputenc}
\usepackage [english, russian] {babel}

\usepackage {authblk}
\usepackage {amsmath}
\usepackage {amsfonts}
\usepackage {amssymb}

\title{Пространственно локализованные решения уравнения Гросса--Питаевского с переодически модулированной нелинейностью}
\author{Лебедев М. Е.}
\affil{Институт математики с вычислительным центром \\ УФИЦ РАН}
\date{Декабрь \\ 2020}

\begin{document}

\maketitle
\thispagestyle{empty} % remove page number

В докладе пойдёт речь об уравнении
\begin{equation}
i \Psi_t + \Psi_{xx} - U(x) \Psi + P(x) |\Psi|^2 \Psi = 0.
\label{eq:init}
\end{equation}
Уравнения такого типа возникают в задачах оптики и теории конденсата Бозе--Эйнштейна (БЭК).
В контексте теории БЭК такое уравнение, называемое также {\it уравнением Гросса--Питаевского}, описывает поведение конденсата в пространственно одномерном случае (``сигарообразный'' конденсат).
Здесь $\Psi(x, t)$ -- макроскопическая волновая функция конденсата, $U(x)$ соответствует потенциалу ловушки, удерживающей конденсат, а функция $P(x)$, называемая также нелинейным потенциалом, описывает характер межатомных взаимодействий.
% Конденсат Бозе--Эйнштейна был предсказан в 20-е годы \textsc{XX} века и впервые получен экспериментально в 1995 году.
% К настоящему времени был сделан большой прогресс в экспериментальной науке, в частности стали возможны эксперименты, в которых, используя явление так называемого резонанса Фешбаха, характерная длина межатомных взаимодействий частиц конденсата может изменяться в пространстве.
% В одномерном случае (``сигарообразный'' конденсат) поведение такого конденсата описывается уравнением Гросса--Питаевского, нелинейное слагаемое в котором имеет дополнительный коэффициент, зависящий от пространственной координаты.

В докладе рассматривается влияние периодического нелинейного потенциала $P(x)$ на структуру семейства стационарных локализованных решений вида $\Psi(x, t) = u(x) e^{i \omega t}$ уравнения \eqref{eq:init} и их устойчивость.
Доказаны некоторые общие утверждения о существовании такого семейства.
Для модельной задачи, когда влиянием линейного потенциала ловушки можно пренебречь, $U(x) \approx 0$, предложен подход, который позволяет детально описать множество стационарных локализованных решений в терминах {\it символической динамики}, а также указаны границы его применимости.
% Оказывается, что при некоторых допущениях каждому локализованному решению можно сопоставить последовательность символов над бесконечным алфавитом.

% Детально рассмотрен случай, когда когда влиянием линейного потенциала ловушки можно пренебречь $U(x) \approx 0$, при этом предложен подход, который позволяет детально описать множество стационарных локализованных решений, а также указаны границы его применимости.
Для случая классической потенциальной ямы $U(x) \sim x^2$ показано, что присутствие периодического нелинейного потенциала может приводить к возникновению решений, которые не имеют аналогов в традиционном уравнении Гросса--Питаевского с постоянным коэффициентом $P(x) \equiv const$ при нелинейности.
Также продемонстрировано, что нелинейные потенциалы такого рода могут служить инструментом стабилизации локализованных решений, которые неустойчивы в случае классической знакопостоянной нелинейности.

\end{document}
