% TODO: Labels in the beginning, not in the end.
\chapter{General Propositions on Regular and Singular Solutions for Stationary States Equation $u_{xx} + Q(x) u + P(x) u^3 = 0$}

\section{Objectives}

In this chapter we formulate general statements about singular and regular solutions for the equation
\begin{equation}
	u_{xx} + Q(x) u + P(x) u^3 = 0.
	\label{eq:main}
\end{equation}
% TODO: Move this part to the introduction.
%As it was shown previously this equation follows from the original Gross--Pitaevskii equation by the the substitution
%\begin{equation}
%	\Psi(t, x) = u(x) e^{-i \omega t},
%\end{equation}
%where the corresponding localization conditions allow us to consider $u(x)$ as a real-valued function.
In general, we suppose that $Q(x), P(x) \in C^1(\mathbb{R})$ and will impose additional restrictions further when it's needed.
% TODO: Definitions of "regular / singular solution" and "collapse" must be presented in introduction.
Mainly we address two questions: (A) when do exist regular solutions of \eqref{eq:main}; (B) what are the conditions for the functions $Q(x)$ and $P(x)$ which can guarantee the existence of the singular solutions for equation \eqref{eq:main}; and (C) what is the behaviour of the collapsing solutions near the collapse point.
In this chapter partial answer to the question (A) is given by the Proposition~\ref{prop:absense-of-singular-solutions}.
On other hand Propositions~\ref{prop:singular-families} and \ref{prop:all-solutions-are-singular} give a partial answer to the question (B).
In particular Proposition~\ref{prop:singular-families} states that if the function $P(x)$ is negative at a point $x = x_0$ then there exist two one-parametric families of solutions collapsing at $x_0$.
Proposition~\ref{prop:singular-families} determines an asymptotic behaviour of these singular solutions families, which gives an answer to the question (C) for these families.

\section{Non-existence of Singular Solutions: $P(x) \ge P_0 > 0$}

This section contains a sufficient condition for non-existence of singular solutions for equation \eqref{eq:main}.
It's given by the following proposition.

\begin{proposition}
	Let functions $Q(x), P(x) \in C^1(\mathbb{R})$, moreover:
	\begin{enumerate}
		\item[(a)] $P(x) \ge P_0 > 0$, $|P'(x)| \le \widetilde{P}$;
		\item[(b)] $Q(x) \ge Q_0$, $|Q'(x)| \le \widetilde{Q}$;
	\end{enumerate}
	then solution of the Cauchy problem for equation \eqref{eq:main} with arbitrary initial conditions $u(x_0) = u_0$, $u_{x}(x_0) = u_0'$ can be continued to the whole real axis $\mathbb{R}$.
	\label{prop:absense-of-singular-solutions}
\end{proposition}
\begin{proof}
	By the existence and uniqueness theorem for ODE there exists an interval $[x_0; x_1)$ such that the solution of the Cauchy problem $u(x)$ for equation \eqref{eq:main} with initial conditions $u(x_0) = u_0$, $u_{x}(x_0) = u_0'$ exists and is unique on this interval, and $u(x) \in C^2[x_0; x_1)$.
	Suppose that $[x_0; x_1)$ is the maximum interval for existence of $u(x)$.
	It means that solution of the Cauchy problem $u(x)$ cannot be continued beyond the point $x_1$.
	Multiplying the original equation by $4u_{x}(x)$ and integrating it over $[x_0, x)$, $x < x_1$, we have the following relation:
	\begin{equation}
	\begin{aligned}
		& 2 u_x^2(x)) + 2 Q(x) u^2(x) - 2 \int \limits_{x_0}^{x} Q'(\xi) u^2(\xi) d\xi + P(x) u^4(x) - \int \limits_{x_0}^x P'(\xi) u^4(\xi) d\xi = \\
		& \quad = 2 (u_0')^2 + 2 Q(x_0) u_0^2 + P(x_0) u_0^4 \equiv C.
		\label{eq:aux-04}
	\end{aligned}	
	\end{equation}
	Omit the term $u_{x}^2(x) \ge 0$ in the left-hand side of the equality, and take into account the lower limits for $Q(x)$, $P(x)$ given by conditions (a), (b).
	Then we arrive at the following inequality:
	\begin{equation}
		2 Q_0 u^2(x) + P_0 u^4(x) \le C + 2 \int \limits_{x_0}^{x} Q'(\xi) u^2(\xi) d\xi + \int \limits_{x_0}^{x} P'(\xi) u^4(\xi) d\xi.
	\end{equation}
	Replace the derivatives $Q'(\xi)$ and $P'(\xi)$ with their upper bounds: $Q'(\xi) \le \widetilde{Q}$, $P'(\xi) \le \widetilde{P}$, where $\widetilde{Q} \ge 0$, $\widetilde{P} \ge 0$.
	Multiplying both sides of the inequality by $P_0$, we have
	\begin{equation}
		2 Q_0 P_0 u^2(x) + P_0^2 u^4(x) \le P_0 C + 2 P_0 \widetilde{Q} \int \limits_{x_0}^{x} u^2(\xi) d\xi + P_0 \widetilde{P} \int \limits_{x_0}^{x} u^4(\xi) d\xi.
		\label{eq:aux-01}
	\end{equation}
	Let $v(x) = (P_0 u^2(x) + Q_0)^2$, $v(x) \ge 0$, substituting this into \eqref{eq:aux-01} gives
	\begin{equation}
		v(x) \le \widetilde{C} + \dfrac{\widetilde{P}}{P_0} \int \limits_{x_0}^{x} w(v(\xi)) d\xi.
		\label{eq:aux-02}
	\end{equation}
	Here $\widetilde{C} = P_0 C + Q_0^2 \ge 0$, $\alpha = 2 \widetilde{Q} P_0 / \widetilde{P} \ge 0$, and $w(v)$ is defined by
	\begin{equation}
		w(v) \equiv \alpha (\sqrt{v} - Q_0) + (\sqrt{v} - Q_0)^2.
	\end{equation}
	Consider the function
	\begin{equation}
		G(s) = \int \limits_{s_0}^{s} \dfrac{dv}{w(v)}.
	\end{equation}
	Here $s_0 > Q_0^2$ is an arbitrary constant, $s \ge s_0$.
	Since $w(v)$ is a positive and monotonically decreasing function, and the integral
	\begin{equation}
		\int \limits_{s_0}^{+\infty} \dfrac{dv}{w(v)}
	\end{equation}
	diverges, one can conclude that $G(s)$ is a positive, monotonically increasing, and unbounded function.
	It means that inverse function $G^{-1}(r)$ is well-defined for $r \ge 0$, increases monotonically, and is unbounded.
	The above-mentioned statements allow as to apply {\it Bihary} inequality \cite[theorem 2.3.1]{Pachpatte} to \eqref{eq:aux-02}.
	This results in the inequality
	\begin{equation}
		v(x) \le G^{-1} \left( G(\widetilde{C}) + \dfrac{\widetilde{P}}{P_0} \int \limits_{x_0}^{x} d\xi \right) = G^{-1} \left( G(\widetilde{C}) \dfrac{\widetilde{P}}{P_0} (x - x_0) \right) < \infty.
		\label{eq:bihari}
	\end{equation}
	Inequality \eqref{eq:bihari} is valid for all $x \in [x_0; x_1)$.
	It follows from \eqref{eq:bihari} that function $v(x)$ is bounded on the whole interval $[x_0; x_1)$:
	\begin{equation}
		v(x) \le M = G^{-1} \left( G(\widetilde{C}) + \dfrac{\widetilde{P}}{P_0} (x_1 - x_0) \right).
	\end{equation}
	We observe that $\widetilde{C} \ge Q_0^2$, moreover $\widetilde{C} = Q_0^2$ only if $u_0 = u_0' = 0$.
	It means that $G(s)$ is well-defined for each constant $\widetilde{C}$ corresponding to any non-zero solution $u(x)$.
	The boundedness of $v(x)$ yields that solution $u(x)$ is also bounded on the segment $[x_0; x_1)$:
	\begin{equation}
		|u(x)| \le \sqrt{\dfrac{\sqrt{M} - Q_0}{P_0}}, \quad x \in [x_0; x_1).
		\label{eq:aux-03}
	\end{equation}
	Substitution of \eqref{eq:aux-03} into \eqref{eq:aux-04} gives the upper bound for the derivative $u_{x}(x)$ on the interval $x \in [x_0; x_1)$.
	Since functions $u(x)$ and $u_{x}(x)$ are continuous and bounded on $[x_0; x_1)$, the values $u(x_1) = u_1$ and $u_{x}(x_1) = u_1'$ are finite.
	Hence there exists a continuation of the solution to the Cauchy problem with the initial conditions $u(x_0) = u_0$, $u_{x}(x_0) = u_0'$ on a larger interval beyond the initial $[x_0; x_1)$.
	It contradicts the original assumption.
	
	Thus we have proved that the solution can be continued for $x > x_0$.
	In order to prove the same statement for $x < x_0$, one can make a substitution $x \to -x$ and employ the same reasoning.
\end{proof}

\begin{corollary}
	If the conditions (a) and (b) are satisfied not on the whole real axis $\mathbb{R}$, but only on some interval $[x_1; x_2]$, then a solution of the Cauchy problem for equation \eqref{eq:main} with arbitrary initial conditions does not collapse at any point of the segment $[x_1; x_2]$.
\end{corollary}

\section{Asymptotic Behaviour at a Collapse Point: $P(x_0) < 0$}

\subsection{Asymptotic Expansions}

If $P(x)$ is negative at least at one point $x_0 \in \mathbb{R}$, formal asymptotic expansions predict existence of two one-parametric families of the solutions for the equation \eqref{eq:main} collapsing at this point.

Let us construct these asymptotic expansions.
We suppose that $P(x_0) = -1$ (this condition can be achieved by a simple renormalisation of the independent variable), denote $\eta = x - x_0$, and assume that in the vicinity of the point $x = x_0$, the following expansions are valid:
\begin{equation}
	Q(x) = Q_0 + Q_1 \eta + Q_2 \eta^2 \dots, \quad P(x) = -1 + P_1 \eta + P_2 \eta^2 + \dots.
\end{equation}
Substituting these expansions into \eqref{eq:main}, we have
\begin{equation}
	u_{\eta\eta} + (Q_0 + Q_1 \eta + Q_2 \eta^2 \dots)u + (-1 + P_1 \eta + P_2 \eta^2 + \dots) u^3 = 0.
	\label{eq:aux-11}
\end{equation}
If a solution $u(\eta)$ of equation \eqref{eq:aux-11} collapses at the point $\eta = 0$ then $u(\eta) \to \pm \infty$, when $\eta \to 0$.
Let $\eta$ approach zero {\it from the right}, $\eta > 0$.
The change $v(\eta) = \eta u(\eta)$, $\eta = e^{-t}$ gives
\begin{equation}
	v_{tt} + 3v_{t} + 2v + e^{-2t} Q(t) v + P(t) v^3 = 0.
	\label{eq:asymptotic-problem}
\end{equation}
Determine the main term of the expansion by balancing $2v$ and $-v^3$ terms.
We have
\begin{equation}
	V_0(t) = \pm \sqrt{2}.
	\label{eq:main-term}
\end{equation}
Now let's define the first order term $V_1(t)$, $v(t) = \pm \sqrt{2} + V_1(t) + o(V_1(t))$.
Substituting the last expression into \eqref{eq:asymptotic-problem}, taking into account the expansions for the functions $Q(t)$, $P(t)$, and omitting the terms of order higher than $e^{-t}$, we obtain
\begin{equation}
	V_{1, tt} + 3V_{1, t} - 4V_1 = \mp 2 \sqrt{2} e^{-t},
\end{equation}
that gives $V_1(t) = \pm \frac{\sqrt{2}}{3} e^{-t}$.
Second, third, and forth order terms $V_n$, $n = 2, 3, 4$, can be found in a similar manner.
For each term the corresponding equation takes form:
\begin{equation}
	V_{n, tt} + 3V_{n, t} - 4V_n = C_n e^{-nt}.
	\label{eq:high-order-terms-equation}
\end{equation}
For $n = 2, 3$ solutions of equation \eqref{eq:high-order-terms-equation} are of the form $V_n \sim e^{-nt}$.
However in the case $n = 4$ the exponent degree in the right hand side coincides with one of the roots of the characteristic polynomial for the differential operator in the left-hand side.
In this case solution of equation \eqref{eq:high-order-terms-equation} must be chosen in the form $Ce^{-4t} - A_3 t e^{-4t}$.
Here $C$ in an arbitrary constant, while $A_3$ can be determined uniquely from the coefficients of the series expansions for $Q(t)$, $P(t)$.
If constant $C$ is fixed, at the further steps of this procedure the corresponding equations are uniquely solvable.
One can note that switching of $+$ to $-$ in the expression \eqref{eq:main-term} leads to the corresponding change of signs for all coefficients $A_n$, $n = 0, 1,...$, that is natural due to the invariance of equation \eqref{eq:main} with respect to the change $u \to -u$.
We have
\begin{equation}
	\pm v(t) = \sqrt{2} + A_0 e^{-t} + A_1 e^{-2t} + A_2 e^{-3t} + A_3 \cdot (-t) \cdot e^{-4t} + C e^{-4t} + \dots.
	\label{eq:expansion-intermediate}
\end{equation}
Explicit expressions for $A_0, \dots, A_3$ are:
\begin{eqnarray}
	& A_0 & = \dfrac{\sqrt{2}}{3} P_1; \label{eq:aux-A0} \\
	& A_1 & = \dfrac{\sqrt{2}}{3} P_2 + \dfrac{\sqrt{2}}{6} Q_0 + \dfrac{2 \sqrt{2}}{9} P_1^2; \label{eq:aux-A1} \\
	& A_2 & = \frac{2\sqrt{2}} 3P_2 P_1 + \frac{7\sqrt{2}}{27} P_1^3 + \frac{\sqrt{2}} 6Q_0 P_1 + \frac{\sqrt{2}} 4Q_1 + \frac{\sqrt{2}} 2P_3; \label{eq:aux-A2} \\
	& A_3 & = -\dfrac{\sqrt{2}}{6} Q_1 P_1 - \dfrac{\sqrt{2}}{5} Q_2 - \dfrac{32 \sqrt{2}}{45} P_2 P_1^2 - \dfrac{3 \sqrt{2}}{5} P_3 P_1 - \\
	&& - \dfrac{2 \sqrt{2}}{15} P_2 Q_0 -\dfrac{2 \sqrt{2}}{15} Q_0 P_1^2 - \dfrac{2 \sqrt{2}}{5} P_4 - \dfrac{28 \sqrt{2}}{135} P_1^4 - \dfrac{4 \sqrt{2}}{15} P_2^2. \label{eq:aux-A3}
\end{eqnarray}

In the other case when $\eta \to 0$ {\it from the left}, $\eta < 0$, similar expansions can be constructed by mean of changes of variables $v(\eta) = \eta u(\eta)$, $\eta = -e^{-t}$.
Expressions for the coefficient $A_n$ remain the same as for $\eta > 0$.

Finally we get an asymptotic expansion for the original solution $u(x)$ for $x \to x_0 \pm 0$:
\begin{equation}
	\pm u(x) = \dfrac{\sqrt{2}}{\eta} + A_0 + A_1 \eta + A_2 \eta^2 + A_3 \eta^3 \ln |\eta| + C \eta^3+ A_4 \eta^4 \ln |\eta| + \dots.
	\label{eq:expansion}
\end{equation}
Here $\eta = x - x_0$, $A_0, \dots, A_3$ are determined by equations \eqref{eq:aux-A0}-\eqref{eq:aux-A3}, and all other coefficients $A_n$, $n > 3$ can be expressed through $Q_n$, $P_n$ and arbitrary constant $C$.

Summarizing all the above mentioned, one can say that asymptotic expansion \eqref{eq:expansion} {\it predicts the existence} of two one-parametric families of solutions collapsing at the point $x_0$.
These families are connected by the symmetry $u \to -u$.
When $x \to x_0$, the solutions from one of these families tend to $+\infty$, while the solutions from another family tend to $-\infty$ correspondingly.

\subsection{Existence of One-Parametric Families of Collapsing Solutions}

Strictly speaking, formal asymptotic expansions \eqref{eq:expansion} do not imply the existence of one-parametric families of solutions collapsing at point $x_0$.
However, the following rigorous statement holds.

\begin{proposition}
	Let $\Omega$ be a neighbourhood of the point $x_0$, $Q(x) \in C^3(\Omega)$ and $P(x) \in C^4(\Omega)$.
	Then there exist two $C^1$-smooth one-parametric families of solutions for the equation \eqref{eq:main} corresponding to expansions \eqref{eq:expansion}, collapsing at the point $x = x_0$ (while approaching from the left, $x < x_0$), and connected by a symmetry $u \to -u$.
	Each of these families can be parametrized by a free variable $C \in \mathbb{R}$ from the expansions \eqref{eq:expansion}.
\label{prop:singular-families}
\end{proposition}
\begin{proof}
	Due to the condition of proposition the following expansions are valid:
	\begin{eqnarray}
		& Q(x) & = Q_0 + Q_1 \eta + Q_2 \eta^2 + \widetilde{Q}(\eta) \eta^3; \\
		& P(x) & = -1 + P_1 \eta + P_2 \eta^2 + P_3 \eta^3 + P_4 \eta^4 + \widetilde{P}(\eta) \eta^5.
	\end{eqnarray}
	Here $\eta = x - x_0$, and $\widetilde{Q}, \widetilde{P} \in C(\Omega)$.
	To prove existence of the family that corresponds to the $+$ sign in \eqref{eq:expansion}	 we introduce the function $z(\eta)$ as follows:
	\begin{equation}
		u(x) = \dfrac{\sqrt{2}}{\eta} + A_0 + A_1 \eta + A_2 \eta^2 + A_3 \eta^3 \ln(-\eta) + z(\eta) \eta^3,
		\label{eq:aux-20}
	\end{equation}
	($z(\eta)$ is a new unknown function).
	Coefficients $A_0, \dots, A_3$ are chosen accordingly to the expressions \eqref{eq:aux-A0}-\eqref{eq:aux-A3}, so the coefficients at the terms $\eta^{-2}$, $\eta^{-1}$, $\eta^0$, and $\eta$ vanish.
	It's easy to check that direct substitution of the \eqref{eq:aux-20} into \eqref{eq:main} yields
	\begin{equation}
		z_{\eta\eta} + \dfrac{6}{\eta} z_{\eta} + g(\eta, z) = 0,
		\label{eq:aux-z}
	\end{equation}
	where $g(\eta, z)$ is a third order polynomial with respect to $z$, and $g(\eta, z) \sim \frac{\ln(-\eta)}{\eta}$ when $\eta \to -0$ and $z$ is fixed.
	The change of variable $\eta = -e^{-t}$ maps the point $\eta = 0$ into $t = +\infty$, and transforms equation \eqref{eq:aux-z} into
	\begin{equation}
		z_{tt} - 5z_t - f(t, z) = 0.
		\label{eq:aux-zt}
	\end{equation}
	Here $f(t, z) \sim t e^{-t}$ while $t \to +\infty$.
	Properties of the function $f(t, z)$ allows us to apply {\it Lemma on Bounded Solutions} from Appendix \ref{appendix:lemma-on-bounded-solutions} to equation \eqref{eq:aux-zt}.
	This lemma states that for $t \to +\infty$ all bounded solutions of equation \eqref{eq:aux-zt} tend to some constant $C$ when $t \to +\infty$, moreover for each $C \in \mathbb{R}$ there exists a unique solution that approaches to that constant asymptotically while $t \to +\infty$.
	Furthermore these solutions form a $C^1$-smooth family.
	Finally, we can return to previous equation \eqref{eq:aux-z}, and then to \eqref{eq:main} to get the initial statement of the proposition.
	The existence of the second family of solutions corresponding to the sign ``$-$'' in \eqref{eq:expansion} follows from the invariance of equation \eqref{eq:main} under the symmetry $u \to -u$.
\end{proof}

Similar one-parametric families of collapsing solutions exist from the right side of the point $x = x_0$.
The corresponding proof can be performed in the same way.

\section{All Solutions Are Singular: $P(x) \le P_0 < 0$, $Q(x) \le Q_0 < 0$}

It turns out that under some assumptions all non-trivial solutions of the equation \eqref{eq:main} are singular.
\begin{proposition}
\label{prop:all-solutions-are-singular}
	Let for $x \in \mathbb{R}$ the conditions $P(x) \le P_0 < 0$, $Q(x) \le Q_0 < 0$ take place.
	Then all solutions of equation \eqref{eq:main} are singular except for the zero one.
\end{proposition}

To prove this proposition we prove the following auxiliary lemma first.
\begin{lemma}
	Let $p, q > 0$ are real constants, then all solutions of equation
	\begin{equation}
		v_{xx} - q v - p v^3 = 0,
		\label{eq:aux-lemma}
	\end{equation}
	are singular except for the zero one.
\end{lemma}
\begin{proof}
	The solution of the Cauchy problem for equation \eqref{eq:aux-lemma} with initial conditions $v(x_0) = v_0$, $v_x(x_0) = v_0'$ can be written in an implicit form as follows:
	\begin{equation}
		\pm \int \limits_{v_0}^{v} \dfrac{d\xi}{\sqrt{C + q \xi^2 + \dfrac{p}{2} \xi^4}} = x - x_0;	\quad C \equiv (v_0')^2 - q v_0^2 - \dfrac{p}{2} v_0^4.
		\label{eq:aux-21}
	\end{equation}
	Choice of the sign in the left hand-side depends on the initial conditions and the value of $x$.
	Integral in the left hand-side of the equality \eqref{eq:aux-21} converges when $v \to \infty$, and hence there exist a value $x^*$,
	\begin{equation}
		x^* = x_0 \int \limits_{v_0}^{\infty} \dfrac{d\xi}{\sqrt{C + q \xi^2 + \dfrac{p}{2} \xi^4}},
	\end{equation}
	such that $v(x)$ goes to infinity while $x$ approaches to the $x^*$ 
	So a solution $v(x)$ with arbitrary non-zero initial conditions is singular, lemma is proved.
\end{proof}

Now we can prove the Proposition 3.
\begin{proof}[Proof of the Proposition 3.]
	We use a so-called {\it Comparison Lemma} from \cite[Appendix C]{AlfimovZezyulin}.
	Consider the equation
	\begin{equation}
		v_{xx} + Q_0 v + P_0 v^3 = 0.
	\end{equation}
	We introduce the notations
	\begin{eqnarray}
		& g(x, \xi) & = -Q(x) \xi - P(x) \xi^3; \\
		& f(x, \xi) = f(\xi) & = -Q_0 \xi - P_0 \xi^3.
	\end{eqnarray}
	Now we apply Comparison Lemma to the following pair of equations:
	\begin{eqnarray}
		&& u_{xx} = g(x, u) \label{eq:comparison-u}; \\
		&& v_{xx} = f(x, v) \label{eq:comparison-v}.
	\end{eqnarray}
	In the domain $D_+ = \{ x \in \mathbb{R}, \xi \in (0; +\infty) \}$ we have $f(x, \xi) \le g(x, \xi)$.
	Let $\widetilde{u}(x)$ be a solution of the Cauchy problem for equation \eqref{eq:comparison-u} with initial conditions $u(x_0) = u_0$, $u_x(x_0) = u_0'$.
	Chose the initial conditions for the Cauchy problem for equation \eqref{eq:comparison-v} as follows: $v(x_0) = u(x_0) = u_0$, $v_x(x_0) = u_x(x_0) = u_0'$; let $\widetilde{v}(x)$ be a solution for that problem.
	Let $u_0 > 0$, then one of the two cases takes place.
	\begin{enumerate}
		\item[(i)] $u_0' \ge 0$.
		Function $\widetilde{v}(x)$ increases monotonically; this fact can be easily established from the phase portrait of equation \eqref{eq:comparison-v}.
		Solution $\widetilde{u}(x)$ bounds the solution $\widetilde{v}(x)$ from above.
		But $\widetilde{v}(x)$ is singular.
		Then it follows from Comparison Lemma that solution $\widetilde{u}(x)$ is also singular.
		\item[(ii)] $u_0' < 0$.
		We make a change of variable $\widetilde{x} = -x$.
		In that case solution $\widetilde{v}(\widetilde{x})$ also increases monotonically, and since $\widetilde{u}(\widetilde{x})$ limits $\widetilde{v}(\widetilde{x})$ from above, $\widetilde{u}(\widetilde{x})$ is singular by Comparison Lemma, hence $\widetilde{u}(x)$ is also singular.
	\end{enumerate}
	Similarly in the domain $D_- = \{ x \in \mathbb{R}, \xi \in (-\infty; 0) \}$, the inequality $f(x, \xi) \ge g(x, \xi)$ holds.
	One can prove in the same manner that in the domain $D_-$ solution $u(x)$ is also singular.
\end{proof}

\section{Summary}

Our main findings on regular and singular solutions for the stationary states equation \eqref{eq:main} are summarised in Table \ref{tab:first-chapter-results}.
Our further findings are focused on the case when $P(x)$ changes its sign.
In the next chapter we describe a so-called {\it method of excluding of singular solutions} which allows us to classify all regular solutions of equation \eqref{eq:main} within the symbolic dynamics framework.

\begin{table}[h!]
	\centering
	\begin{tabular}{ | p{4cm} | l || p{10cm} | }
		\hline
		$P(x)$ & $Q(x)$ & \\
		\hline
		$P(x) > 0$ & --- & All the solutions can be continued to the whole real line, singular solutions do not exist (Proposition~\ref{prop:absense-of-singular-solutions}). \\
		\hline
		$P(x) < 0$ at least at one point $x = x_0$ & --- & There exists a pair of one-parametrical families of solutions collapsing at point $x = x_0$ and related by the symmetry $u \to -u$ (Proposition~\ref{prop:singular-families}). \\
		\hline
		$P(x) < 0$ & $Q(x) < 0$ & All solutions are singular except for the zero one (Proposition~\ref{prop:all-solutions-are-singular}). \\
		\hline
		$P(x)$ changes sign along $\mathbb{R}$ & --- & Singular solutions are generic.
		That fact allows to apply the so-called {\it method of excluding of singular solutions} and classify all regular solutions in terms of symbolic dynamics.
		We describe this method and its application in Chapter~\ref{chapter:II}. \\
		\hline
	\end{tabular}
	\caption{
		Summary of the results for the Chapter 1.
		The results of this Chapter were published in \cite{AlfimovLebedev}.
	}
	\label{tab:first-chapter-results}
\end{table}
